\begin{longtable}[c]{@{}lllll@{}}
    \caption{Previous Studies}
    \label{tab:my-table}\\
    \toprule
    Author(year) & Research Purpose & Considered Variables \\* \midrule
    \endfirsthead
    %
    \endhead
    %
    \bottomrule
    \endfoot
    %
    \endlastfoot
    %
    Moon et al (2021) & \parbox[t]{5cm}{Identifying the Potential Vertiport Location for Addressing Traffic in the Busan Area\citep{MoonShiKang2021}} & \parbox[t]{5cm}{\raggedright Transportation connectivity, Workplace density, Official land price, Number of commuters, Estimated income quintile, Average transportation cost, Parking lot accessibility, Residential density, Noise\\} \\
    Sung et al (2023) & \parbox[t]{5cm}{Optimal Location Selection and Efficiency Analysis of Driving Route Travel Time for Practical Use of Vertiport\citep{SungKimChoiCho2023}} & \parbox[t]{5cm}{\raggedright Airspace restriction area, Natural environment conservation area, river, Public transportation demand, Living population density, Average annual income, Individual official land price, Transportation connection, Residential density\\} \\
    Jeong and Hwang (2021) & \parbox[t]{5cm}{Analysis of demand for eVTOL to reduce commuting time in the Seoul metropolitan area and selection of vertiport location\citep{JeongSoHwang2021}} & \parbox[t]{5cm}{\raggedright Commuting population, Green belt, Airspace restriction area, Residential area\\} \\
    Kim et al (2023) & \parbox[t]{5cm}{Suggesting the direction of vertiport construction to increase usage through analysis in terms of usage environment\citep{kim-2023}\\} & \parbox[t]{5cm}{\raggedright Accessibility, Affordability, Living environment\\}  \\
    Jung et al (2021) & \parbox[t]{5cm}{Deriving Vertiport Location Selection Factors and Analysis of Location Assessment Factors \citep{JungYuYun2021}\\} & \parbox[t]{5cm}{\raggedright Land costs, Transportation connectivity, Obstacles, Ease of supply and construction of power sources, Noise environment, Law\\} \\
    Son et al (2023) & \parbox[t]{5cm}{Selecting the location of the vertiport in the Seoul metropolitan area, building a network, estimating usage, and calculating the scale of the vertiport \citep{Son-2023}\\} & \parbox[t]{5cm}{\raggedright Traffic volume} \\
    
    Min et al (2020) & \parbox[t]{5cm}{Select a vertiport construction area in Seoul by analyzing the composition of physical infrastructure and considerations by type \citep{Min-2020}\\} & \parbox[t]{5cm}{\raggedright Airspace restriction area in Seoul, Floating population, Connection of ground transportation} \\* \bottomrule
\end{longtable}
